% !TeX encoding = UTF-8
% !TeX program = pdflatex
% !BIB program = bibtex

%%% To write an article in English, please use the option ``english'' in order
%%% to get the correct hyphenation patterns and terms.
%%% \documentclass[english]{class}
%%% for anonymizing an article you can use the ``anonymous'' option.
%%%
%%% Um einen Artikel auf deutsch zu schreiben, genügt es die Klasse ohne
%%% Parameter zu laden.
%%% Zur Anonymisierung kann die ``anonymous'' Option genutzt werden.
\documentclass[]{lni}
%%
\begin{document}
%%% Mehrere Autoren werden durch \and voneinander getrennt.
%%% Die Fußnote enthält die Adresse sowie eine E-Mail-Adresse.
%%% Das optionale Argument (sofern angegeben) wird für die Kopfzeile verwendet.
\title[Ein Kurztitel]{Ein sehr langer Titel über mehrere Zeilen mit sehr vielen
Worten und noch mehr Buchstaben}
%% \subtitle{Untertitel / Subtitle} % if needed
\author[1]{Vorname1 Nachname1}{vorname.name@affiliation1.de}{0000-0000-0000-0000}
\author[2]{Firstname2 Lastname2}{vorname.name@affiliation2.de}{0000-0000-0000-0000}
\author[1]{Firstname3 Lastname3}{vorname.name@affiliation1.de}{0000-0000-0000-0000}
\author[1]{Firstname4 Lastname4}{vorname.name@affiliation1.de}{0000-0000-0000-0000}%
\affil[1]{Universität\\Abteilung\\Straße\\Postleitzahl Ort\\Land}
\affil[2]{University\\Department\\Address\\Country}
\maketitle

\begin{abstract}
Dies ist eine kurze Übersicht über das Dokument mit einer Länge von
70 bis 150 Wörtern. Es sollte ein Absatz sein, der die relevantesten
Aspekte enthält.
\end{abstract}
\begin{keywords}
Schlagwort1 \and Schlagwort2 %Keyword1 \and Keyword2
\end{keywords}
%%% Beginn des Artikeltexts
\section{Überschrift/Heading}

%%% Angabe der .bib-Datei (ohne Endung) / State .bib file (im Falle der Nutzung von BibTeX)
%% \bibliography{mybibfile}
%% \printbibliography % im Falle der Nutzung von biblatex
\end{document}
