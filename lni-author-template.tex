% !TeX encoding = UTF-8
% !TeX program = pdflatex
% !BIB program = bibtex

%%% Um einen Artikel auf deutsch zu schreiben, genügt es die Klasse ohne
%%% Parameter zu laden.
%%% Zur Anonymisierung kann die ``anonymous'' Option genutzt werden.
\documentclass[]{lni}
%%% To write an article in English, please use the option ``english'' in order
%%% to get the correct hyphenation patterns and terms.
%%% \documentclass[english]{class}
%%% for anonymizing an article you can use the ``anonymous'' option.
%%
\begin{document}
%%% Mehrere Autoren werden durch \and voneinander getrennt.
%%% Die Fußnote enthält die Adresse sowie eine E-Mail-Adresse.
%%% Das optionale Argument (sofern angegeben) wird für die Kopfzeile verwendet.
\title[Ein Kurztitel]{Ein sehr langer Titel über mehrere Zeilen mit sehr vielen
Worten und noch mehr Buchstaben}
%% \subtitle{Untertitel / Subtitle} % if needed
\author[Vorname1 Nachname1 \and Firstname2 Lastname2]
{Vorname1 Nachname1\footnote{Universität, Abteilung, Straße, Postleitzahl Ort,
Land \email{emailaddress@author1}} \and
Firstname2 Lastname2\footnote{University, Department, Address, Country
\email{emailaddress@author2}}}
%% (en) numbering starts at this number
%% (de) Beginn der Seitenzählung für diesen Beitrag
\startpage{11}
\editor{Herausgeber et al.} % Names of Editors
\booktitle{Name-der-Konferenz} % Name of book title
\yearofpublication{2017}
%% \lnidoi{18.18420/provided-by-editor-02} % activate and adapt if the DOI is known
%% \setcctype{by-sa} % can be: zero, by, by-sa, by-nc, by-nd, by-nc-sa, by-nc-nd
\maketitle

\begin{abstract}
Dies ist eine kurze Übersicht über das Dokument mit einer Länge von
70 bis 150 Wörtern. Es sollte ein Absatz sein, der die relevantesten
Aspekte enthält.
\end{abstract}
\begin{keywords}
Schlagwort1 \and Schlagwort2 %Keyword1 \and Keyword2
\end{keywords}
%%% Beginn des Artikeltexts
\section{Überschrift/Heading}

%%% Angabe der .bib-Datei (ohne Endung) / State .bib file (im Falle der Nutzung von BibTeX)
%% \bibliography{mybibfile}
%% \printbibliography % im Falle der Nutzung von biblatex
\end{document}
