% !TeX encoding = UTF-8
% !TeX spellcheck = de_DE

\usepackage{iftex}\ifluatex\else
\pdfoutput=1
\fi

%% Dies gibt Warnungen aus, sollten veraltete LaTeX-Befehle verwendet werden
\RequirePackage[l2tabu, orthodox]{nag}

%% Zur Anonymisierung kann die ``anonymous'' Option genutzt werden, siehe auch \anon-Makro.
\documentclass[utf8,biblatex]{lni}
\bibliography{lni-paper-example-de}

%% Schöne Tabellen mittels \toprule, \midrule, \bottomrule
\usepackage{booktabs}

%% Zu Demonstrationszwecken
\usepackage{amsmath}
\usepackage[math]{blindtext}
\usepackage{mwe}

%% BibLaTeX-Sonderkonfiguration,
%% falls man schnell eine existierende Bibliographie wiederverwenden will, aber nicht die .bib-Datei händisch anpassen möchte.
%% Bitte \iffalse und \fi entfernen, dann ist diese Konfiguration aktiviert.

\iffalse
\AtEveryBibitem{%
  \ifentrytype{article}{%
  }{%
    \clearfield{doi}%
    \clearfield{issn}%
    \clearfield{url}%
    \clearfield{urldate}%
  }%
  \ifentrytype{inproceedings}{%
  }{%
    \clearfield{doi}%
    \clearfield{issn}%
    \clearfield{url}%
    \clearfield{urldate}%
  }%
}
\fi

\begin{document}
%%% Mehrere Autoren werden durch \and voneinander getrennt.
%%% Die Fußnote enthält die Adresse sowie eine E-Mail-Adresse.
%%% Das optionale Argument (sofern angegeben) wird für die Kopfzeile verwendet.
\title[Ein Kurztitel]{Ein sehr langer Titel über mehrere Zeilen mit sehr vielen Worten und noch mehr Buchstaben}
%%%\subtitle{Untertitel / Subtitle} % falls benötigt
\author[Vorname1 Nachname1 \and Vorname2 Nachname2]
{Vorname1 Nachname1\footnote{Universität, Abteilung, Straße, Postleitzahl Ort, Land \email{emailaddress@author1}} \and
 Vorname2 Nachname2\footnote{University, Department, Address, Country \email{emailaddress@author2}}}
\startpage{11} % Beginn der Seitenzählung für diesen Beitrag
\editor{Herausgeber et al.}    % Namen der Herausgeber
\booktitle{Name-der-Konferenz} % Name des Tagungsband; optional Kurztitel
\yearofpublication{2017}
%%%\lnidoi{18.18420/provided-by-editor-02} % Falls bekannt
%%%\setcctype{by-sa} % Möglichkeiten: zero, by, by-sa, by-nc, by-nd, by-nc-sa, by-nc-nd
\maketitle

\begin{abstract}
Die \LaTeX-Klasse \texttt{lni} setzt die Layout-Vorgaben für Beiträge in LNI Konferenzbänden um.
Dieses Dokument beschreibt ihre Verwendung und ist ein Beispiel für die entsprechende Darstellung.
Der Abstract ist ein kurzer Überblick über die Arbeit der zwischen 70 und 150 Wörtern lang sein und das Wichtigste enthalten sollte.
Die Formatierung erfolgt automatisch innerhalb des abstract-Bereichs.
\end{abstract}

\begin{keywords}
LNI Guidelines \and \LaTeX Vorlage
\end{keywords}

\section{Verwendung}
Die GI gibt unter \url{http://www.gi-ev.de/LNI} Vorgaben für die Formatierung von Dokumenten in der LNI Reihe.
Für \LaTeX-Dokumente werden diese durch die Dokumentenklasse \texttt{lni} realisiert.

Dieses Dokument basiert auf der offiziellen Dokumentation, simplifiziert und setzt grundlegendes LaTeX-Wissen voraus.
Es werden generische Platzhalter an die entsprechenden Stellen (wie beispielsweise die Authoren-Angaben) gesetzt und nicht weiter an anderer Stelle dokumentiert.

Dieses Template ist wie folgt gegliedert:
\Cref{sec:demos} zeigt Demonstrationen der LNI-Verlage.
\Cref{sec:lniconformance} zeigt die Einhaltung der Richtlinien durch einfachen Text.

\section{Demonstrationen}
\label{sec:demos}
Das Symbol für Potenzmengen ($\powerset$) wird korrekt angezeigt.
Es ist kein Weierstraß-p ($\wp$) mehr.

Spitze Klammen können direkt eingegeben werden: <test />

Anonymisierungen können mittels anonymous-Option in der documentclass automatisch vorgenommen werden. Dafür gibt es das \texttt{anon}-Makro, z.\,B. \anon{Geheim für Review} und \anon[nur für Review]{nur für finale Version}.

Hier eine kleine Demonstration von \href{https://www.ctan.org/pkg/microtype}{microtype}:
\blindtext

\section{Demonstration der Einhaltung der Richtlinien}
\label{sec:lniconformance}

\subsection{Literaturverzeichnis}
Der letzte Abschnitt zeigt ein beispielhaftes Literaturverzeichnis für Bücher mit einem Autor \cite{Ez10} und zwei AutorInnen \cite{AB00}, einem Beitrag in Proceedings mit drei AutorInnen \cite{ABC01}, einem Beitrag in einem LNI Band mit mehr als drei AutorInnen \cite{Az09}, zwei Bücher mit den jeweils selben vier AutorInnen im selben Erscheinungsjahr \cite{Wa14} und \cite{Wa14b}, ein Journal \cite{Gl06}, eine Website \cite{GI19} bzw.\ anderweitige Literatur ohne konkrete AutorInnenschaft \cite{XX14}.
Es wird biblatex verwendet, da es UTF8 sauber unterstützt und \href{https://github.com/gi-ev/LNI/issues/5}{im Gegensatz zu lni.bst} keine Fehler beim bibtexen auftreten.

Referenzen sollten nicht direkt als Subjekt eingebunden werden, sondern immer nur durch Authorenanganben:
Beispiel: \Citet{AB00} geben ein Beispiel, aber auch \citet{Az09}.
Hinweis: Großes C bei \texttt{Citet}, wenn es am Satzanfang steht. Dies ist analog zu \texttt{Cref}.

Formatierung und Abkürzungen werden für die Referenzen \texttt{book}, \texttt{inbook}, \texttt{proceedings}, \texttt{inproceedings}, \texttt{article}, \texttt{online} und \texttt{misc} automatisch vorgenommen.
Mögliche Felder für Referenzen können der Beispieldatei \texttt{lni-paper-example-de.bib} entnommen werden.
Andere Referenzen sowie Felder müssen allenfalls nachträglich angepasst werden.

\subsection{Abbildungen}
\Cref{fig:demo} zeigt eine Abbildung.

\begin{figure}
  \centering
  \includegraphics[width=.8\textwidth]{example-image}
  \caption{Demographik}
  \label{fig:demo}
\end{figure}

\subsection{Tabellen}
\Cref{tab:demo} zeigt eine Tabelle.

\begin{table}
\centering
\begin{tabular}{lll}
\toprule
Überschriftsebenen & Beispiel & Schriftgröße und -art \\
\midrule
Titel (linksbündig) & Der Titel \ldots & 14 pt, Fett\\
Überschrift 1 & 1 Einleitung & 12 pt, Fett\\
Überschrift 2 & 2.1 Titel & 10 pt, Fett\\
\bottomrule
\end{tabular}
\caption{Die Überschriftsarten}
\label{tab:demo}
\end{table}

\subsection{Programmcode}
Die LNI-Formatvorlage verlangt die Einrückung von Listings vom linken Rand.
In der \texttt{lni}-Dokumentenklasse ist dies für die \texttt{verbatim}-Umgebung realisiert.

\begin{verbatim}
public class Hello {
    public static void main (String[] args) {
        System.out.println("Hello World!");
    }
}
\end{verbatim}

Alternativ kann auch die \texttt{lstlisting}-Umgebung verwendet werden.

\Cref{L1} zeigt uns ein Beispiel, das mit Hilfe der \texttt{lstlisting}-Umgebung realisiert ist.

\begin{lstlisting}[caption={Beschreibung}, label=L1, language=Java]
public class Hello {
    public static void main (String[] args) {
        System.out.println("Hello World!");
    }
}
\end{lstlisting}

\subsection{Formeln und Gleichungen}

Die korrekte Einrückung und Nummerierung für Formeln ist bei den Umgebungen
\texttt{equation} und \texttt{align} gewährleistet.

\begin{equation}
  1=4-3
\end{equation}
und
\begin{align}
  2&=7-5\\
  3&=2-1
\end{align}

%% \bibliography{lni-paper-example-de.tex} ist hier nicht erlaubt: biblatex erwartet dies bei der Preambel
%% Starten Sie "biber paper", um eine Biliographie zu erzeugen.
\printbibliography

\end{document}
